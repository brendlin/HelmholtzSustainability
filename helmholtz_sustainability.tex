\documentclass[11pt]{article}

\usepackage[a4paper,left=1.0in,top=1in,right=1.0in,bottom=1in,nofoot,footskip=1cm]{geometry}

\usepackage{xspace}
\usepackage{soul} %% Wrapped underline with \ul
\usepackage{hyperref}

%% \usepackage{lineno}
%% \linenumbers

\title{Active Demand-side Energy Management at DESY\\ 
       {\large A proposal for supporting a renewable, sustainable, and stable local energy economy}}
\author{Kurt Brendlinger}
\date{}

\renewcommand{\refname}{\normalfont\selectfont\normalsize References:}

\begin{document}
\maketitle
\thispagestyle{empty}

Climate change mitigation is perhaps the most important existential challenges facing society today.
%With its dedication to sustainability, DESY has a responsibility to 
As one of the largest consumers of electricity in the Hamburg region, DESY has both a
responsibility and a unique opportunity
to shape the future of the energy system in which it participates, to promote a cleaner, more
sustainable electrical grid system.

In 2018, DESY consumed roughly 150 GWh of electricity,
about half of which is used by the DESY accelerator facilities
(PETRA, FLASH, etc.); around 30\% is consumed for cryogenics,
and only about 5\% for DESY data centers
\cite{jensen}.
With a typical consumption of about 21 MW (about the magnitude of a mid-size wind park),
DESY has an opportunity to engage in the with the local electricity system, and to adapt its
operations to the realities of a renewables-dominated electrical grid. 

In Germany today, renewable
energy accounts for roughly 45\% of power production, which is dominated by wind and solar power,
both intermittent sources. Germany has dedicated itself to an aggressive timeline to phase out the
remaining fossil-fuel-based energy sources by 2050; however, reaching this goal will only lead to
even more intermittency of the electricity supply.

%, and the energy storage capacity required to smooth out these economically challenging conditions.
%% Sustainable energy economy.

The electricity grid and its operators face \textbf{four main challenges}
due to the increasing use of renewable energies:
first, because renewable energies cannot be controlled, periods of \textbf{peak demand}
or insufficient renewables must be met using conventional power plants (so-called ``peaker plants'')
that emit CO$_{2}$. Likewise, an over-production of renewable energies when demand is too low
can lead to operators requiring renewables to shut down (\textbf{curtailment}), effectively
wasting free electrical energy.

Third, geographical disparities between generation and electrical load can cause
\textbf{grid congestion}, whereby there is insufficient transmission capacity to transport the
electricity where it is needed and the transmission cables risk exceeding their
capacities. Mitigating this effect is costly, requiring transmission system operators to pay
electricity generators located in areas with higher (lower) loads to produce more (less) energy.
This process, called \textit{redispatching}, is especially expensive because one electricity
provider must be paid to generate more energy, while the other must be compensated for lost
revenue, effectively doubling the energy cost.
%
Finally, while weather forecasting is improving significantly, misses in forecasting
can lead to unforeseen \textbf{schedule deviations} of renewables production, again requiring
costly and CO$_2$-producing redispatching remedies. As conventional power plants become a
diminished part of the energy mix, these four challenges pose a significant threat to
the stability of the electrical grid.

%% Local electricity transmission and distribution operators face three challenges 

As a big power consumer in a major metropolitan area, DESY has the opportunity to work with the
local energy transmission and distribution network operators to help increase grid stability,
and avoid costly grid scenarios. By identifying significant flexibilities in its energy use,
DESY can use a technique called \textbf{demand response} to help mitigate all four of these
challenges. By pausing accelerator activities, delaying computing resource use,
or allowing cooling temperatures to rise by a few degrees, for time scales on the order of 15
minutes to an hour and coordinated with grid system operators, DESY can take an active role in
achieving a more sustainable energy grid.

Because demand response and flexible loads can help
solve these problems at low capital cost, they can lower the overall costs of the grid system by
reducing the need for CO$_2$-emitting reserve capacity power plants or costly storage solutions.
Because of this, demand
response or demand-side management services are typically reimbursed by transmission system
operators, which would serve as a revenue stream for DESY and would potentially offset any
expenditures in setting up and managing the programme.

Developing an initiative for electricity demand response at DESY would require a concerted effort
and support from multiple departments within the DESY campus.
As a first step, DESY would need to identify the flexible loads available at its on-site facilities,
and engage in a dialogue about balancing the the needs of DESY researchers, external users and
other stakeholders using these resources.
DESY would also need to approach transmission and distribution system operators in the Hamburg
region, and understand the economic and legal framework within which the project would operate.
Finally, DESY would need to expand its energy monitoring systems and develop the software,
on-site infrastructure, and communication capabilities for implementing such a system.

With the development of a system of demand-side management, DESY would not only support the
stability, security, and sustainability of the local electrical grid in Hamburg, but it could
serve as a model by which other public and private institutions, many of which are significant
energy consumers operating in metropolitan areas,
could also implement such a model. By exporting such a system to other institutes and industries,
DESY could have a far-reaching impact on the resiliency of the entire electrical grid, helping to
achieve a greater renewable energy penetration faster and at lower overall system costs.
By actively participating in the electricity grid on which we rely, DESY can help to create a
cleaner, greener, and more sustainable future.

%% So for instance by pausing our computer clusters we
%% could remove stress on the grid caused by unexpected renewable forecast deviations.

%%  - Demand response or demand-side management.
%%  - Reduces the cost of the overall energy system by reducing the need for peaking power plants that use natural gas or other CO$_2$-emitting fuels.
%%  - DESY has a responsibility to adapt its operations to the realities of a renewables-dominated electrical grid...?
%%  - DESY has a responsibility to engage in the with the local electricity system, and to adapt its
%% operations to the realities of a renewables-dominated electrical grid.
%%  - Schedule deviations

%% Offering computing clusters as a "Flex load" (mentioned in my last email) to help with renewable
%% energy variability issues. In principle system operators would actually pay to have this
%% flexibility, so it is potentially a source of revenue for DESY.


 %% - Balance the needs of DESY collaboration and external researchers using the DESY on-site faciliies.
 %% - Develop a model by which other major scientific facilities can implement similar 

\raggedright

\bibliographystyle{hunsrt}
%\bibliographystyle{natbib}
\bibliography{helmholtz_sustainability}

\end{document}
